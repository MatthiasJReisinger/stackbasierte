%%%%%%%%%%%%%%%%%%%%%%%%%%%%%%%%%%%%%%%%%%%%%
\begin{frame}{Context
% Overview%Main Contribution
}
\ms

% \bi

% \item[] \cemph{Gentzen-type refutation systems} for two basic \cemph{3-valued logics}.
% \smallskip
\bi
\item The traditional view about proof calculi is that they are \cemph{assertional}---their aim is to axiomatise the \cemph{valid propositions} of a logic.\pause

\ms
\item But we can also have a complementary view:

\bi
\item Instead of axiomatising the valid sentences we may axiomatise the \cemph{invalid} ones.\pause

\item In such a system, false propositions are deduced from other (elementary) false ones.\pause

\ei

\ms
\resitem Calculi axiomatising the invalid sentences of a logic are called \cemph{rejection systems} or \cemph{complementary calculi}.%\pause


\ei
\end{frame}
