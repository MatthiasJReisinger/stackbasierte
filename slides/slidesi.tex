%%%%%%%%%%%%%%%%%%%%%%%%%%%%%%%%%%%%%%%%%%%%%
\usetikzlibrary{arrows}

\begin{frame}{Einf\"uhrung}
\bi
\item Wir betrachten \cemph{modallogische Formeln}, aufgebaut durch 
\bi
	\item \cemph{Atome} $\mathcal{P} = \{p_1,p_2,p_3,\ldots\}$,
	\item un\"are Junktoren $\neg, \Box, \lozenge$,
	\item bin\"are Junktoren $\impl, \wedge, \vee$.
\ei
\ms
\item \blue{Beispiele:} 
\bi
 \item $\Box p_1 \impl p_1$, 
 \item $\neg p_2 \vee \Box\lozenge p_1$,
 \item $\Box p_1 \impl \Box \Box p_1$.
\ei
\ei
\end{frame}

\begin{frame}{Einf\"uhrung (fgs.)}
Eine \cemph{Kripke-Interpretation} (Interpretation) ist ein Tripel $\mathcal{M} = \langle W, R, v\rangle$, wobei\pause
\bi
	\item $W$ eine nicht-leere Menge ist (die Menge der \cemph{m\"oglichen Welten}),\pause
	\item $R \subseteq W \times W$,\pause
	\item $v : W \times \mathcal{P} \rightarrow \{0,1\}$.
\ei

\bs
\pause

\ithand\ Interpretationen definieren Graphen!
\end{frame}

\begin{frame}{Einf\"uhrung (fgs.)}
Sei $\mathcal{M} = \langle W, R, v\rangle$ eine Interpretation und $w \in W$. Wir definieren $\mathcal{M}, w \models \varphi$ f\"ur alle Formeln $\varphi$ wie folgt:
\bi
	\item $\mathcal{M}, w \models p$ $\iff$ $v(w,p) = 1$ f\"ur ein Atom $p$.
	\item $\mathcal{M}, w \models \varphi_1 \wedge \varphi_2$ $\iff$ $\mathcal{M}, w \models \varphi_1$ und $\mathcal{M}, w \models \varphi_2$.
	\item $\mathcal{M}, w \models \varphi_1 \vee \varphi_2$ $\iff$ $\mathcal{M}, w \models \varphi_1$ oder $\mathcal{M}, w \models \varphi_2$.
	\item $\mathcal{M}, w \models \varphi_1 \impl \varphi_2$ $\iff$ $\mathcal{M}, w \not\models \varphi_1$ oder $\mathcal{M}, w \models \varphi_2$.
	\item $\mathcal{M}, w \models \neg\varphi$ $\iff$ $\mathcal{M}, w \not\models \varphi$.
	\item $\mathcal{M}, w \models \Box\varphi$ $\iff$ $\mathcal{M}, w \models \varphi$ \red{f\"ur alle Welten $w'$ mit $wRw'$}.
	\item $\mathcal{M}, w \models \lozenge\varphi$ $\iff$ $\mathcal{M}, w \models \varphi$ \red{f\"ur mindestens eine Welt $w'$ mit $wRw'$}.
	
\ei
\end{frame}

\begin{frame}{Einf\"uhrung (fgs.)}
\bi
\item {\bf Aufgabenstellung:} Gegeben eine modallogische Formel $\varphi$, eine (endliche) Interpretation $\mathcal{M} = \langle W, R, v \rangle$, sowie eine Welt $w \in W$, \"uberpr\"ufe ob $\mathcal{M},w \models \varphi$.
\ei
\end{frame}

\begin{frame}{Beispiel}
\begin{tabular}{llp{2em}|p{2em}ll}
$1$ & $\textcolor<8->{magenta}{q}$ & & $2$ & $\textcolor<5>{magenta}{p}$ \\
\hline
$3$ & $\textcolor<8->{magenta}{q}$ & & $4$ & $\textcolor<5>{magenta}{p}$
\end{tabular}
\qquad\qquad Gilt $\Box (p \wedge \lozenge q)$ in Welt $1$?

\begin{center}
\begin{tikzpicture}[->,>=stealth',shorten >=1pt,auto,node distance=4cm,
  thick,main node/.style={circle,fill=blue!20,draw}]

  \node[main node] (1) [onslide={<2,7,8>{draw=magenta, ultra thick}}] {$1$};
  \node[main node] (2) [below left of=1, onslide={<3,4,5,6>{draw=magenta, ultra thick}}] {$2$};
  \node[main node] (3) [below right of=2, onslide={<7,8>{draw=magenta, ultra thick}}] {$3$};
  \node[main node] (4) [below right of=1, onslide={<3,4,5,6>{draw=magenta, ultra thick}}] {$4$};

  \path[every node/.style={font=\sffamily\small}]
    (1) edge [bend right, onslide=<3>{magenta,ultra thick}] node[left] {} (2)
    (2) edge [<->] node {} (4)
        edge [loop left] node {} (2)
        edge [bend right,onslide=<7>{magenta,ultra thick}] node[left] {} (3)
    (3) edge [bend right] node[right] {} (4)
    (4) edge [loop right] node {} (4)
        edge [bend right,<->, onslide=<3>{magenta,ultra thick},onslide=<7>{magenta,ultra thick}] node[right] {} (1);

  \uncover<2->{\node (5) [below = .1cm of 1, onslide=<2>{magenta}] {$\Box (p \wedge \lozenge q)$\only<7->{, $\textcolor<7->{magenta}{q}$}};}
  \uncover<3-3>{\node (6) [above right = .05cm of 2,onslide=<3>{magenta}] {$p \wedge \lozenge q$};}
  \uncover<3-3>{\node (7) [above left = .05cm of 4,onslide=<3>{magenta}] {$p \wedge \lozenge q$};}
  \uncover<4->{\node (8) [above right = .05cm of 2,onslide=<4>{magenta}] {$\textcolor<5>{magenta}{p}$, $\textcolor<6>{magenta}{\lozenge q}$};}
  \uncover<4->{\node (9) [above left = .05cm of 4,onslide=<4>{magenta}] {$\textcolor<5>{magenta}{p}$, $\textcolor<6>{magenta}{\lozenge q}$};}
  \uncover<7->{\node (10) [above = .05cm of 3,onslide={<7,8>{magenta}}] {$q$};}
\end{tikzpicture}
\end{center}
\end{frame}

\begin{frame}[c,allowframebreaks]
\begin{center}
\Huge
\structure{Demo \& Code}
\end{center}
\end{frame}